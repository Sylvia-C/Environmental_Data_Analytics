\documentclass[]{article}
\usepackage{lmodern}
\usepackage{amssymb,amsmath}
\usepackage{ifxetex,ifluatex}
\usepackage{fixltx2e} % provides \textsubscript
\ifnum 0\ifxetex 1\fi\ifluatex 1\fi=0 % if pdftex
  \usepackage[T1]{fontenc}
  \usepackage[utf8]{inputenc}
\else % if luatex or xelatex
  \ifxetex
    \usepackage{mathspec}
  \else
    \usepackage{fontspec}
  \fi
  \defaultfontfeatures{Ligatures=TeX,Scale=MatchLowercase}
\fi
% use upquote if available, for straight quotes in verbatim environments
\IfFileExists{upquote.sty}{\usepackage{upquote}}{}
% use microtype if available
\IfFileExists{microtype.sty}{%
\usepackage{microtype}
\UseMicrotypeSet[protrusion]{basicmath} % disable protrusion for tt fonts
}{}
\usepackage[margin=2.54cm]{geometry}
\usepackage{hyperref}
\hypersetup{unicode=true,
            pdftitle={Assignment 4: Data Wrangling},
            pdfauthor={Siying Chen},
            pdfborder={0 0 0},
            breaklinks=true}
\urlstyle{same}  % don't use monospace font for urls
\usepackage{color}
\usepackage{fancyvrb}
\newcommand{\VerbBar}{|}
\newcommand{\VERB}{\Verb[commandchars=\\\{\}]}
\DefineVerbatimEnvironment{Highlighting}{Verbatim}{commandchars=\\\{\}}
% Add ',fontsize=\small' for more characters per line
\usepackage{framed}
\definecolor{shadecolor}{RGB}{248,248,248}
\newenvironment{Shaded}{\begin{snugshade}}{\end{snugshade}}
\newcommand{\KeywordTok}[1]{\textcolor[rgb]{0.13,0.29,0.53}{\textbf{#1}}}
\newcommand{\DataTypeTok}[1]{\textcolor[rgb]{0.13,0.29,0.53}{#1}}
\newcommand{\DecValTok}[1]{\textcolor[rgb]{0.00,0.00,0.81}{#1}}
\newcommand{\BaseNTok}[1]{\textcolor[rgb]{0.00,0.00,0.81}{#1}}
\newcommand{\FloatTok}[1]{\textcolor[rgb]{0.00,0.00,0.81}{#1}}
\newcommand{\ConstantTok}[1]{\textcolor[rgb]{0.00,0.00,0.00}{#1}}
\newcommand{\CharTok}[1]{\textcolor[rgb]{0.31,0.60,0.02}{#1}}
\newcommand{\SpecialCharTok}[1]{\textcolor[rgb]{0.00,0.00,0.00}{#1}}
\newcommand{\StringTok}[1]{\textcolor[rgb]{0.31,0.60,0.02}{#1}}
\newcommand{\VerbatimStringTok}[1]{\textcolor[rgb]{0.31,0.60,0.02}{#1}}
\newcommand{\SpecialStringTok}[1]{\textcolor[rgb]{0.31,0.60,0.02}{#1}}
\newcommand{\ImportTok}[1]{#1}
\newcommand{\CommentTok}[1]{\textcolor[rgb]{0.56,0.35,0.01}{\textit{#1}}}
\newcommand{\DocumentationTok}[1]{\textcolor[rgb]{0.56,0.35,0.01}{\textbf{\textit{#1}}}}
\newcommand{\AnnotationTok}[1]{\textcolor[rgb]{0.56,0.35,0.01}{\textbf{\textit{#1}}}}
\newcommand{\CommentVarTok}[1]{\textcolor[rgb]{0.56,0.35,0.01}{\textbf{\textit{#1}}}}
\newcommand{\OtherTok}[1]{\textcolor[rgb]{0.56,0.35,0.01}{#1}}
\newcommand{\FunctionTok}[1]{\textcolor[rgb]{0.00,0.00,0.00}{#1}}
\newcommand{\VariableTok}[1]{\textcolor[rgb]{0.00,0.00,0.00}{#1}}
\newcommand{\ControlFlowTok}[1]{\textcolor[rgb]{0.13,0.29,0.53}{\textbf{#1}}}
\newcommand{\OperatorTok}[1]{\textcolor[rgb]{0.81,0.36,0.00}{\textbf{#1}}}
\newcommand{\BuiltInTok}[1]{#1}
\newcommand{\ExtensionTok}[1]{#1}
\newcommand{\PreprocessorTok}[1]{\textcolor[rgb]{0.56,0.35,0.01}{\textit{#1}}}
\newcommand{\AttributeTok}[1]{\textcolor[rgb]{0.77,0.63,0.00}{#1}}
\newcommand{\RegionMarkerTok}[1]{#1}
\newcommand{\InformationTok}[1]{\textcolor[rgb]{0.56,0.35,0.01}{\textbf{\textit{#1}}}}
\newcommand{\WarningTok}[1]{\textcolor[rgb]{0.56,0.35,0.01}{\textbf{\textit{#1}}}}
\newcommand{\AlertTok}[1]{\textcolor[rgb]{0.94,0.16,0.16}{#1}}
\newcommand{\ErrorTok}[1]{\textcolor[rgb]{0.64,0.00,0.00}{\textbf{#1}}}
\newcommand{\NormalTok}[1]{#1}
\usepackage{graphicx,grffile}
\makeatletter
\def\maxwidth{\ifdim\Gin@nat@width>\linewidth\linewidth\else\Gin@nat@width\fi}
\def\maxheight{\ifdim\Gin@nat@height>\textheight\textheight\else\Gin@nat@height\fi}
\makeatother
% Scale images if necessary, so that they will not overflow the page
% margins by default, and it is still possible to overwrite the defaults
% using explicit options in \includegraphics[width, height, ...]{}
\setkeys{Gin}{width=\maxwidth,height=\maxheight,keepaspectratio}
\IfFileExists{parskip.sty}{%
\usepackage{parskip}
}{% else
\setlength{\parindent}{0pt}
\setlength{\parskip}{6pt plus 2pt minus 1pt}
}
\setlength{\emergencystretch}{3em}  % prevent overfull lines
\providecommand{\tightlist}{%
  \setlength{\itemsep}{0pt}\setlength{\parskip}{0pt}}
\setcounter{secnumdepth}{0}
% Redefines (sub)paragraphs to behave more like sections
\ifx\paragraph\undefined\else
\let\oldparagraph\paragraph
\renewcommand{\paragraph}[1]{\oldparagraph{#1}\mbox{}}
\fi
\ifx\subparagraph\undefined\else
\let\oldsubparagraph\subparagraph
\renewcommand{\subparagraph}[1]{\oldsubparagraph{#1}\mbox{}}
\fi

%%% Use protect on footnotes to avoid problems with footnotes in titles
\let\rmarkdownfootnote\footnote%
\def\footnote{\protect\rmarkdownfootnote}

%%% Change title format to be more compact
\usepackage{titling}

% Create subtitle command for use in maketitle
\newcommand{\subtitle}[1]{
  \posttitle{
    \begin{center}\large#1\end{center}
    }
}

\setlength{\droptitle}{-2em}

  \title{Assignment 4: Data Wrangling}
    \pretitle{\vspace{\droptitle}\centering\huge}
  \posttitle{\par}
    \author{Siying Chen}
    \preauthor{\centering\large\emph}
  \postauthor{\par}
    \date{}
    \predate{}\postdate{}
  

\begin{document}
\maketitle

\subsection{OVERVIEW}\label{overview}

This exercise accompanies the lessons in Environmental Data Analytics
(ENV872L) on data wrangling.

\subsection{Directions}\label{directions}

\begin{enumerate}
\def\labelenumi{\arabic{enumi}.}
\tightlist
\item
  Change ``Student Name'' on line 3 (above) with your name.
\item
  Use the lesson as a guide. It contains code that can be modified to
  complete the assignment.
\item
  Work through the steps, \textbf{creating code and output} that fulfill
  each instruction.
\item
  Be sure to \textbf{answer the questions} in this assignment document.
  Space for your answers is provided in this document and is indicated
  by the ``\textgreater{}'' character. If you need a second paragraph be
  sure to start the first line with ``\textgreater{}''. You should
  notice that the answer is highlighted in green by RStudio.
\item
  When you have completed the assignment, \textbf{Knit} the text and
  code into a single PDF file. You will need to have the correct
  software installed to do this (see Software Installation Guide) Press
  the \texttt{Knit} button in the RStudio scripting panel. This will
  save the PDF output in your Assignments folder.
\item
  After Knitting, please submit the completed exercise (PDF file) to the
  dropbox in Sakai. Please add your last name into the file name (e.g.,
  ``Salk\_A04\_DataWrangling.pdf'') prior to submission.
\end{enumerate}

The completed exercise is due on Thursday, 7 February, 2019 before class
begins.

\subsection{Set up your session}\label{set-up-your-session}

\begin{enumerate}
\def\labelenumi{\arabic{enumi}.}
\item
  Check your working directory, load the \texttt{tidyverse} package, and
  upload all four raw data files associated with the EPA Air dataset.
  See the README file for the EPA air datasets for more information
  (especially if you have not worked with air quality data previously).
\item
  Generate a few lines of code to get to know your datasets (basic data
  summaries, etc.).
\end{enumerate}

\begin{Shaded}
\begin{Highlighting}[]
\CommentTok{#1}
\CommentTok{# Check working directory}
\KeywordTok{getwd}\NormalTok{()}
\end{Highlighting}
\end{Shaded}

\begin{verbatim}
## [1] "/Users/Sylvia/Downloads/ENV872/ENV872"
\end{verbatim}

\begin{Shaded}
\begin{Highlighting}[]
\CommentTok{# Load package}
\KeywordTok{library}\NormalTok{(tidyverse)}
\end{Highlighting}
\end{Shaded}

\begin{verbatim}
## -- Attaching packages -------------------------------------------------------------------------- tidyverse 1.2.1 --
\end{verbatim}

\begin{verbatim}
## v ggplot2 3.1.0     v purrr   0.2.5
## v tibble  2.0.1     v dplyr   0.7.8
## v tidyr   0.8.2     v stringr 1.3.1
## v readr   1.3.1     v forcats 0.3.0
\end{verbatim}

\begin{verbatim}
## -- Conflicts ----------------------------------------------------------------------------- tidyverse_conflicts() --
## x dplyr::filter() masks stats::filter()
## x dplyr::lag()    masks stats::lag()
\end{verbatim}

\begin{Shaded}
\begin{Highlighting}[]
\CommentTok{# Import EPA air dataset}
\NormalTok{EPA_O3_}\DecValTok{17}\NormalTok{ <-}\StringTok{ }\KeywordTok{read.csv}\NormalTok{(}\StringTok{"./Data/Raw/EPAair_O3_NC2017_raw.csv"}\NormalTok{)}
\NormalTok{EPA_O3_}\DecValTok{18}\NormalTok{ <-}\StringTok{ }\KeywordTok{read.csv}\NormalTok{(}\StringTok{"./Data/Raw/EPAair_O3_NC2018_raw.csv"}\NormalTok{)}
\NormalTok{EPA_PM25_}\DecValTok{17}\NormalTok{ <-}\StringTok{ }\KeywordTok{read.csv}\NormalTok{(}\StringTok{"./Data/Raw/EPAair_PM25_NC2017_raw.csv"}\NormalTok{)}
\NormalTok{EPA_PM25_}\DecValTok{18}\NormalTok{ <-}\StringTok{ }\KeywordTok{read.csv}\NormalTok{(}\StringTok{"./Data/Raw/EPAair_PM25_NC2018_raw.csv"}\NormalTok{)}

\CommentTok{#2}
\KeywordTok{head}\NormalTok{(EPA_O3_}\DecValTok{17}\NormalTok{)}
\end{Highlighting}
\end{Shaded}

\begin{verbatim}
##     Date Source   Site.ID POC Daily.Max.8.hour.Ozone.Concentration UNITS
## 1 3/1/17    AQS 370030005   1                                0.041   ppm
## 2 3/2/17    AQS 370030005   1                                0.046   ppm
## 3 3/3/17    AQS 370030005   1                                0.046   ppm
## 4 3/4/17    AQS 370030005   1                                0.046   ppm
## 5 3/5/17    AQS 370030005   1                                0.046   ppm
## 6 3/6/17    AQS 370030005   1                                0.048   ppm
##   DAILY_AQI_VALUE             Site.Name DAILY_OBS_COUNT PERCENT_COMPLETE
## 1              38 Taylorsville Liledoun              17              100
## 2              43 Taylorsville Liledoun              17              100
## 3              43 Taylorsville Liledoun              17              100
## 4              43 Taylorsville Liledoun              17              100
## 5              43 Taylorsville Liledoun              17              100
## 6              44 Taylorsville Liledoun              17              100
##   AQS_PARAMETER_CODE AQS_PARAMETER_DESC CBSA_CODE
## 1              44201              Ozone     25860
## 2              44201              Ozone     25860
## 3              44201              Ozone     25860
## 4              44201              Ozone     25860
## 5              44201              Ozone     25860
## 6              44201              Ozone     25860
##                      CBSA_NAME STATE_CODE          STATE COUNTY_CODE
## 1 Hickory-Lenoir-Morganton, NC         37 North Carolina           3
## 2 Hickory-Lenoir-Morganton, NC         37 North Carolina           3
## 3 Hickory-Lenoir-Morganton, NC         37 North Carolina           3
## 4 Hickory-Lenoir-Morganton, NC         37 North Carolina           3
## 5 Hickory-Lenoir-Morganton, NC         37 North Carolina           3
## 6 Hickory-Lenoir-Morganton, NC         37 North Carolina           3
##      COUNTY SITE_LATITUDE SITE_LONGITUDE
## 1 Alexander       35.9138        -81.191
## 2 Alexander       35.9138        -81.191
## 3 Alexander       35.9138        -81.191
## 4 Alexander       35.9138        -81.191
## 5 Alexander       35.9138        -81.191
## 6 Alexander       35.9138        -81.191
\end{verbatim}

\begin{Shaded}
\begin{Highlighting}[]
\KeywordTok{colnames}\NormalTok{(EPA_O3_}\DecValTok{18}\NormalTok{)}
\end{Highlighting}
\end{Shaded}

\begin{verbatim}
##  [1] "Date"                                
##  [2] "Source"                              
##  [3] "Site.ID"                             
##  [4] "POC"                                 
##  [5] "Daily.Max.8.hour.Ozone.Concentration"
##  [6] "UNITS"                               
##  [7] "DAILY_AQI_VALUE"                     
##  [8] "Site.Name"                           
##  [9] "DAILY_OBS_COUNT"                     
## [10] "PERCENT_COMPLETE"                    
## [11] "AQS_PARAMETER_CODE"                  
## [12] "AQS_PARAMETER_DESC"                  
## [13] "CBSA_CODE"                           
## [14] "CBSA_NAME"                           
## [15] "STATE_CODE"                          
## [16] "STATE"                               
## [17] "COUNTY_CODE"                         
## [18] "COUNTY"                              
## [19] "SITE_LATITUDE"                       
## [20] "SITE_LONGITUDE"
\end{verbatim}

\begin{Shaded}
\begin{Highlighting}[]
\KeywordTok{dim}\NormalTok{(EPA_PM25_}\DecValTok{17}\NormalTok{)}
\end{Highlighting}
\end{Shaded}

\begin{verbatim}
## [1] 9494   20
\end{verbatim}

\begin{Shaded}
\begin{Highlighting}[]
\KeywordTok{summary}\NormalTok{(EPA_PM25_}\DecValTok{18}\NormalTok{)}
\end{Highlighting}
\end{Shaded}

\begin{verbatim}
##       Date         Source        Site.ID               POC       
##  1/26/18:  39   AirNow: 783   Min.   :370110002   Min.   :1.000  
##  2/1/18 :  39   AQS   :6828   1st Qu.:370630015   1st Qu.:3.000  
##  2/19/18:  39                 Median :371190041   Median :3.000  
##  1/14/18:  38                 Mean   :371031969   Mean   :3.011  
##  1/8/18 :  38                 3rd Qu.:371290002   3rd Qu.:3.000  
##  2/7/18 :  38                 Max.   :371830021   Max.   :5.000  
##  (Other):7380                                                    
##  Daily.Mean.PM2.5.Concentration      UNITS      DAILY_AQI_VALUE
##  Min.   :-2.800                 ug/m3 LC:7611   Min.   : 0.00  
##  1st Qu.: 5.000                                 1st Qu.:21.00  
##  Median : 7.200                                 Median :30.00  
##  Mean   : 7.554                                 Mean   :31.03  
##  3rd Qu.: 9.800                                 3rd Qu.:41.00  
##  Max.   :34.200                                 Max.   :97.00  
##                                                                
##                  Site.Name    DAILY_OBS_COUNT PERCENT_COMPLETE
##  Millbrook School     : 621   Min.   :1       Min.   :100     
##  Board Of Ed. Bldg.   : 428   1st Qu.:1       1st Qu.:100     
##  Garinger High School : 421   Median :1       Median :100     
##  Durham Armory        : 415   Mean   :1       Mean   :100     
##  Lexington water tower: 411   3rd Qu.:1       3rd Qu.:100     
##  Pitt Agri. Center    : 409   Max.   :1       Max.   :100     
##  (Other)              :4906                                   
##  AQS_PARAMETER_CODE                              AQS_PARAMETER_DESC
##  Min.   :88101      Acceptable PM2.5 AQI & Speciation Mass:1246    
##  1st Qu.:88101      PM2.5 - Local Conditions              :6365    
##  Median :88101                                                     
##  Mean   :88167                                                     
##  3rd Qu.:88101                                                     
##  Max.   :88502                                                     
##                                                                    
##    CBSA_CODE                                 CBSA_NAME      STATE_CODE
##  Min.   :11700   Raleigh, NC                      :1274   Min.   :37  
##  1st Qu.:19000   Charlotte-Concord-Gastonia, NC-SC:1171   1st Qu.:37  
##  Median :25860                                    :1025   Median :37  
##  Mean   :30249   Winston-Salem, NC                : 803   Mean   :37  
##  3rd Qu.:39580   Asheville, NC                    : 447   3rd Qu.:37  
##  Max.   :49180   Durham-Chapel Hill, NC           : 415   Max.   :37  
##  NA's   :1025    (Other)                          :2476               
##             STATE       COUNTY_CODE            COUNTY     SITE_LATITUDE  
##  North Carolina:7611   Min.   : 11.0   Mecklenburg:1171   Min.   :34.36  
##                        1st Qu.: 63.0   Wake       : 947   1st Qu.:35.26  
##                        Median :119.0   Buncombe   : 428   Median :35.64  
##                        Mean   :103.2   Durham     : 415   Mean   :35.59  
##                        3rd Qu.:129.0   Davidson   : 411   3rd Qu.:35.87  
##                        Max.   :183.0   Pitt       : 409   Max.   :36.11  
##                                        (Other)    :3830                  
##  SITE_LONGITUDE  
##  Min.   :-83.44  
##  1st Qu.:-80.87  
##  Median :-79.84  
##  Mean   :-79.95  
##  3rd Qu.:-78.57  
##  Max.   :-76.21  
## 
\end{verbatim}

\begin{Shaded}
\begin{Highlighting}[]
\KeywordTok{class}\NormalTok{(EPA_O3_}\DecValTok{18}\OperatorTok{$}\NormalTok{Date)}
\end{Highlighting}
\end{Shaded}

\begin{verbatim}
## [1] "factor"
\end{verbatim}

\subsection{Wrangle individual datasets to create processed
files.}\label{wrangle-individual-datasets-to-create-processed-files.}

\begin{enumerate}
\def\labelenumi{\arabic{enumi}.}
\setcounter{enumi}{2}
\tightlist
\item
  Change date to date
\item
  Select the following columns: Date, DAILY\_AQI\_VALUE, Site.Name,
  AQS\_PARAMETER\_DESC, COUNTY, SITE\_LATITUDE, SITE\_LONGITUDE
\item
  For the PM2.5 datasets, fill all cells in AQS\_PARAMETER\_DESC with
  ``PM2.5'' (all cells in this column should be identical).
\item
  Save all four processed datasets in the Processed folder.
\end{enumerate}

\begin{Shaded}
\begin{Highlighting}[]
\CommentTok{#3}
\NormalTok{EPA_O3_}\DecValTok{17}\OperatorTok{$}\NormalTok{Date <-}\StringTok{ }\KeywordTok{as.Date}\NormalTok{(EPA_O3_}\DecValTok{17}\OperatorTok{$}\NormalTok{Date, }\DataTypeTok{format =} \StringTok{"%m/%d/%y"}\NormalTok{)}
\NormalTok{EPA_O3_}\DecValTok{18}\OperatorTok{$}\NormalTok{Date <-}\StringTok{ }\KeywordTok{as.Date}\NormalTok{(EPA_O3_}\DecValTok{18}\OperatorTok{$}\NormalTok{Date, }\DataTypeTok{format =} \StringTok{"%m/%d/%y"}\NormalTok{)}
\NormalTok{EPA_PM25_}\DecValTok{17}\OperatorTok{$}\NormalTok{Date <-}\StringTok{ }\KeywordTok{as.Date}\NormalTok{(EPA_PM25_}\DecValTok{17}\OperatorTok{$}\NormalTok{Date, }\DataTypeTok{format =} \StringTok{"%m/%d/%y"}\NormalTok{)}
\NormalTok{EPA_PM25_}\DecValTok{18}\OperatorTok{$}\NormalTok{Date <-}\StringTok{ }\KeywordTok{as.Date}\NormalTok{(EPA_PM25_}\DecValTok{18}\OperatorTok{$}\NormalTok{Date, }\DataTypeTok{format =} \StringTok{"%m/%d/%y"}\NormalTok{)}

\CommentTok{#4}
\NormalTok{EPA_O3_}\FloatTok{17.}\NormalTok{select <-}\StringTok{ }\KeywordTok{select}\NormalTok{(EPA_O3_}\DecValTok{17}\NormalTok{, Date, DAILY_AQI_VALUE, Site.Name, AQS_PARAMETER_DESC, COUNTY, SITE_LATITUDE, SITE_LONGITUDE)}
\NormalTok{EPA_O3_}\FloatTok{18.}\NormalTok{select <-}\StringTok{ }\KeywordTok{select}\NormalTok{(EPA_O3_}\DecValTok{18}\NormalTok{, Date, DAILY_AQI_VALUE, Site.Name, AQS_PARAMETER_DESC, COUNTY, SITE_LATITUDE, SITE_LONGITUDE)}
\NormalTok{EPA_PM25_}\FloatTok{17.}\NormalTok{select <-}\StringTok{ }\KeywordTok{select}\NormalTok{(EPA_PM25_}\DecValTok{17}\NormalTok{, Date, DAILY_AQI_VALUE, Site.Name, AQS_PARAMETER_DESC, COUNTY, SITE_LATITUDE, SITE_LONGITUDE)}
\NormalTok{EPA_PM25_}\FloatTok{18.}\NormalTok{select <-}\StringTok{ }\KeywordTok{select}\NormalTok{(EPA_PM25_}\DecValTok{18}\NormalTok{, Date, DAILY_AQI_VALUE, Site.Name, AQS_PARAMETER_DESC, COUNTY, SITE_LATITUDE, SITE_LONGITUDE)}

\CommentTok{#5}
\NormalTok{EPA_PM25_}\FloatTok{17.}\NormalTok{select}\OperatorTok{$}\NormalTok{AQS_PARAMETER_DESC <-}\StringTok{ "PM2.5"}
\NormalTok{EPA_PM25_}\FloatTok{18.}\NormalTok{select}\OperatorTok{$}\NormalTok{AQS_PARAMETER_DESC <-}\StringTok{ "PM2.5"}

\CommentTok{#6}
\KeywordTok{write.csv}\NormalTok{(EPA_O3_}\FloatTok{17.}\NormalTok{select, }\DataTypeTok{row.names =} \OtherTok{FALSE}\NormalTok{, }\DataTypeTok{file =} \StringTok{"./Data/Processed/EPAair_O3_NC2017_Processed.csv"}\NormalTok{)}
\KeywordTok{write.csv}\NormalTok{(EPA_O3_}\FloatTok{18.}\NormalTok{select, }\DataTypeTok{row.names =} \OtherTok{FALSE}\NormalTok{, }\DataTypeTok{file =} \StringTok{"./Data/Processed/EPAair_O3_NC2018_Processed.csv"}\NormalTok{)}
\KeywordTok{write.csv}\NormalTok{(EPA_PM25_}\FloatTok{17.}\NormalTok{select, }\DataTypeTok{row.names =} \OtherTok{FALSE}\NormalTok{, }\DataTypeTok{file =} \StringTok{"./Data/Processed/EPAair_PM25_NC2017_Processed.csv"}\NormalTok{)}
\KeywordTok{write.csv}\NormalTok{(EPA_PM25_}\FloatTok{18.}\NormalTok{select, }\DataTypeTok{row.names =} \OtherTok{FALSE}\NormalTok{, }\DataTypeTok{file =} \StringTok{"./Data/Processed/EPAair_PM25_NC2018_Processed.csv"}\NormalTok{)}
\end{Highlighting}
\end{Shaded}

\subsection{Combine datasets}\label{combine-datasets}

\begin{enumerate}
\def\labelenumi{\arabic{enumi}.}
\setcounter{enumi}{6}
\tightlist
\item
  Combine the four datasets with \texttt{rbind}. Make sure your column
  names are identical prior to running this code.
\item
  Wrangle your new dataset with a pipe function (\%\textgreater{}\%) so
  that it fills the following conditions:
\end{enumerate}

\begin{itemize}
\tightlist
\item
  Sites: Blackstone, Bryson City, Triple Oak
\item
  Add columns for ``Month'' and ``Year'' by parsing your ``Date'' column
  (hint: \texttt{separate} function or \texttt{lubridate} package)
\end{itemize}

\begin{enumerate}
\def\labelenumi{\arabic{enumi}.}
\setcounter{enumi}{8}
\tightlist
\item
  Spread your datasets such that AQI values for ozone and PM2.5 are in
  separate columns. Each location on a specific date should now occupy
  only one row.
\item
  Call up the dimensions of your new tidy dataset.
\item
  Save your processed dataset with the following file name:
  ``EPAair\_O3\_PM25\_NC1718\_Processed.csv''
\end{enumerate}

\begin{Shaded}
\begin{Highlighting}[]
\CommentTok{#7}
\NormalTok{EPAair_combined <-}\StringTok{ }\KeywordTok{rbind}\NormalTok{(EPA_O3_}\FloatTok{17.}\NormalTok{select, EPA_O3_}\FloatTok{18.}\NormalTok{select, EPA_PM25_}\FloatTok{17.}\NormalTok{select, EPA_PM25_}\FloatTok{18.}\NormalTok{select)}

\CommentTok{#8}
\NormalTok{EPAair_combined.filter <-}\StringTok{ }
\StringTok{  }\NormalTok{EPAair_combined }\OperatorTok
\StringTok{  }\KeywordTok{filter}\NormalTok{(Site.Name }\OperatorTok{==}\StringTok{ "Blackstone"} \OperatorTok{|}\StringTok{ }\NormalTok{Site.Name }\OperatorTok{==}\StringTok{ "Bryson City"} \OperatorTok{|}\StringTok{ }\NormalTok{Site.Name }\OperatorTok{==}\StringTok{ "Triple Oak"}\NormalTok{) }\OperatorTok
\StringTok{  }\KeywordTok{separate}\NormalTok{(Date, }\KeywordTok{c}\NormalTok{(}\StringTok{"Y"}\NormalTok{, }\StringTok{"m"}\NormalTok{), }\DataTypeTok{remove =} \OtherTok{FALSE}\NormalTok{)}
\end{Highlighting}
\end{Shaded}

\begin{verbatim}
## Warning: Expected 2 pieces. Additional pieces discarded in 2986 rows [1, 2,
## 3, 4, 5, 6, 7, 8, 9, 10, 11, 12, 13, 14, 15, 16, 17, 18, 19, 20, ...].
\end{verbatim}

\begin{Shaded}
\begin{Highlighting}[]
\CommentTok{#9}
\NormalTok{EPAair_combined.tidy <-}\StringTok{ }\KeywordTok{spread}\NormalTok{(EPAair_combined.filter, AQS_PARAMETER_DESC, DAILY_AQI_VALUE)}

\CommentTok{#10}
\KeywordTok{dim}\NormalTok{(EPAair_combined.tidy)}
\end{Highlighting}
\end{Shaded}

\begin{verbatim}
## [1] 1953    9
\end{verbatim}

\begin{Shaded}
\begin{Highlighting}[]
\CommentTok{#11}
\KeywordTok{write.csv}\NormalTok{(EPAair_combined.tidy, }\DataTypeTok{row.names =} \OtherTok{FALSE}\NormalTok{, }\DataTypeTok{file =} \StringTok{"./Data/Processed/EPAair_O3_PM25_NC1718_Processed.csv.csv"}\NormalTok{)}
\end{Highlighting}
\end{Shaded}

\subsection{Generate summary tables}\label{generate-summary-tables}

\begin{enumerate}
\def\labelenumi{\arabic{enumi}.}
\setcounter{enumi}{11}
\tightlist
\item
  Use the split-apply-combine strategy to generate two new data frames:
\end{enumerate}

\begin{enumerate}
\def\labelenumi{\alph{enumi}.}
\tightlist
\item
  A summary table of mean AQI values for O3 and PM2.5 by month
\item
  A summary table of the mean, minimum, and maximum AQI values of O3 and
  PM2.5 for each site
\end{enumerate}

\begin{enumerate}
\def\labelenumi{\arabic{enumi}.}
\setcounter{enumi}{12}
\tightlist
\item
  Display the data frames.
\end{enumerate}

\begin{Shaded}
\begin{Highlighting}[]
\CommentTok{#12a}
\NormalTok{EPAair_combined.tidy.summaryA <-}\StringTok{ }
\StringTok{  }\NormalTok{EPAair_combined.tidy }\OperatorTok
\StringTok{  }\KeywordTok{group_by}\NormalTok{(m) }\OperatorTok
\StringTok{  }\KeywordTok{filter}\NormalTok{(}\OperatorTok{!}\KeywordTok{is.na}\NormalTok{(Ozone) }\OperatorTok{&}\StringTok{ }\OperatorTok{!}\KeywordTok{is.na}\NormalTok{(PM2.}\DecValTok{5}\NormalTok{)) }\OperatorTok
\StringTok{  }\KeywordTok{summarise}\NormalTok{(}\DataTypeTok{meanO3 =} \KeywordTok{mean}\NormalTok{(Ozone),}
            \DataTypeTok{meanPM25 =} \KeywordTok{mean}\NormalTok{(PM2.}\DecValTok{5}\NormalTok{))}

\CommentTok{#12b}
\NormalTok{EPAair_combined.tidy.summaryB <-}\StringTok{ }
\StringTok{  }\NormalTok{EPAair_combined.tidy }\OperatorTok
\StringTok{  }\KeywordTok{group_by}\NormalTok{(Site.Name) }\OperatorTok
\StringTok{  }\KeywordTok{filter}\NormalTok{(}\OperatorTok{!}\KeywordTok{is.na}\NormalTok{(Ozone) }\OperatorTok{&}\StringTok{ }\OperatorTok{!}\KeywordTok{is.na}\NormalTok{(PM2.}\DecValTok{5}\NormalTok{)) }\OperatorTok
\StringTok{  }\KeywordTok{summarise}\NormalTok{(}\DataTypeTok{meanO3 =} \KeywordTok{mean}\NormalTok{(Ozone),}
            \DataTypeTok{minO3 =} \KeywordTok{min}\NormalTok{(Ozone),}
            \DataTypeTok{maxO3 =} \KeywordTok{max}\NormalTok{(Ozone),}
            \DataTypeTok{meanPM25 =} \KeywordTok{mean}\NormalTok{(PM2.}\DecValTok{5}\NormalTok{),}
            \DataTypeTok{minPM25 =} \KeywordTok{min}\NormalTok{(PM2.}\DecValTok{5}\NormalTok{),}
            \DataTypeTok{maxPM25 =} \KeywordTok{max}\NormalTok{(PM2.}\DecValTok{5}\NormalTok{))}

\CommentTok{#13}
\KeywordTok{print}\NormalTok{(EPAair_combined.tidy.summaryA)}
\end{Highlighting}
\end{Shaded}

\begin{verbatim}
## # A tibble: 12 x 3
##    m     meanO3 meanPM25
##    <chr>  <dbl>    <dbl>
##  1 01      31.5     34.2
##  2 02      35.4     37.6
##  3 03      42.4     37.4
##  4 04      43.5     31.5
##  5 05      39.5     30.6
##  6 06      39.2     30.9
##  7 07      38.3     31.9
##  8 08      34.4     32.3
##  9 09      32.6     30.7
## 10 10      32.3     30.1
## 11 11      30.1     42.1
## 12 12      29.8     46.6
\end{verbatim}

\begin{Shaded}
\begin{Highlighting}[]
\KeywordTok{print}\NormalTok{(EPAair_combined.tidy.summaryB)}
\end{Highlighting}
\end{Shaded}

\begin{verbatim}
## # A tibble: 2 x 7
##   Site.Name   meanO3 minO3 maxO3 meanPM25 minPM25 maxPM25
##   <fct>        <dbl> <dbl> <dbl>    <dbl>   <dbl>   <dbl>
## 1 Blackstone    38.3     8    97     36.7       0      83
## 2 Bryson City   35.4     5    71     30.3       3      68
\end{verbatim}


\end{document}
